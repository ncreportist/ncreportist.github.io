\documentclass[]{article}
\usepackage{lmodern}
\usepackage{amssymb,amsmath}
\usepackage{ifxetex,ifluatex}
\usepackage{fixltx2e} % provides \textsubscript
\ifnum 0\ifxetex 1\fi\ifluatex 1\fi=0 % if pdftex
  \usepackage[T1]{fontenc}
  \usepackage[utf8]{inputenc}
\else % if luatex or xelatex
  \ifxetex
    \usepackage{mathspec}
  \else
    \usepackage{fontspec}
  \fi
  \defaultfontfeatures{Ligatures=TeX,Scale=MatchLowercase}
\fi
% use upquote if available, for straight quotes in verbatim environments
\IfFileExists{upquote.sty}{\usepackage{upquote}}{}
% use microtype if available
\IfFileExists{microtype.sty}{%
\usepackage{microtype}
\UseMicrotypeSet[protrusion]{basicmath} % disable protrusion for tt fonts
}{}
\usepackage[margin=1in]{geometry}
\usepackage{hyperref}
\hypersetup{unicode=true,
            pdftitle={YF Notes: The Partial Kingdom (talk)},
            pdfborder={0 0 0},
            breaklinks=true}
\urlstyle{same}  % don't use monospace font for urls
\usepackage{graphicx,grffile}
\makeatletter
\def\maxwidth{\ifdim\Gin@nat@width>\linewidth\linewidth\else\Gin@nat@width\fi}
\def\maxheight{\ifdim\Gin@nat@height>\textheight\textheight\else\Gin@nat@height\fi}
\makeatother
% Scale images if necessary, so that they will not overflow the page
% margins by default, and it is still possible to overwrite the defaults
% using explicit options in \includegraphics[width, height, ...]{}
\setkeys{Gin}{width=\maxwidth,height=\maxheight,keepaspectratio}
\IfFileExists{parskip.sty}{%
\usepackage{parskip}
}{% else
\setlength{\parindent}{0pt}
\setlength{\parskip}{6pt plus 2pt minus 1pt}
}
\setlength{\emergencystretch}{3em}  % prevent overfull lines
\providecommand{\tightlist}{%
  \setlength{\itemsep}{0pt}\setlength{\parskip}{0pt}}
\setcounter{secnumdepth}{0}
% Redefines (sub)paragraphs to behave more like sections
\ifx\paragraph\undefined\else
\let\oldparagraph\paragraph
\renewcommand{\paragraph}[1]{\oldparagraph{#1}\mbox{}}
\fi
\ifx\subparagraph\undefined\else
\let\oldsubparagraph\subparagraph
\renewcommand{\subparagraph}[1]{\oldsubparagraph{#1}\mbox{}}
\fi

%%% Use protect on footnotes to avoid problems with footnotes in titles
\let\rmarkdownfootnote\footnote%
\def\footnote{\protect\rmarkdownfootnote}

%%% Change title format to be more compact
\usepackage{titling}

% Create subtitle command for use in maketitle
\newcommand{\subtitle}[1]{
  \posttitle{
    \begin{center}\large#1\end{center}
    }
}

\setlength{\droptitle}{-2em}

  \title{YF Notes: The Partial Kingdom (talk)}
    \pretitle{\vspace{\droptitle}\centering\huge}
  \posttitle{\par}
    \author{}
    \preauthor{}\postauthor{}
      \predate{\centering\large\emph}
  \postdate{\par}
    \date{2018-11-04}


\begin{document}
\maketitle

\hypertarget{introduction}{%
\subsection{INTRODUCTION}\label{introduction}}

I've printed off all the passages that I will be referring to so you
don't have to look them up. Try your best to follow along. I'll refer to
the passage references as I go. As a reminder in \textbf{Gen 12} God
made a promise with Abraham. He promises that Abraham would have
decendants, that he would reach the promised land, that there he would
recieve blessing and kings would come from him which would be a blessing
to the nations under them Tonight, we will be covering around 1200 years
spanning from Abraham to Solomon so a lot to squeeze in. We see partial
fulfilment of these promises. The aim is to not get lost in the detail
but rather to see the sweep of the bible story line. I will stop at some
important details which I don't want you to miss.

\hypertarget{gods-people}{%
\subsection{GOD'S PEOPLE}\label{gods-people}}

\hypertarget{abraham-and-isaac}{%
\subsubsection{Abraham and Isaac}\label{abraham-and-isaac}}

It is important to note that there was nothing particularly special
about Abraham that he or his descendants deserved these promises. In
\textbf{Gen 15} it is revealed that Abraham ``believed in the Lord, and
he counted it to him as righteousness''. ``He was accepted by God, not
on the basis of his own Goodness but by faith in the promises of God''
{[}1{]}. We can also tell from this passage that Abraham seems to have a
bit of a problem. He and his wife are very old and have no children to
carry on Abraham's name. Abraham and Sarah decide to take matters into
their own hands and Abraham takes Sarah's maid, Hagar, as a wife who
bears Abraham a son.

In \textbf{Gen 17}, God makes it clear here that His people will not
rise from Hagar's son, but from a son born to Sarah as per the original
plan. By taking Hagar as a wife, Abraham demonstrated that he did not
have trust God's plan. However, true to God's promise, Sarah conceives
and bears Isaac on her old age. When Isaac had grown up, God commands
Abraham to take Isaac up to the mountain and sacrifice him on an altar

This time, if we look in in \textbf{Gen 22}, we see that Abraham
demonstates a change in attitude. He walks Isaac up the mountain with
out protest. ``He knew that the future of the promises depended on
Isaac's survival'' {[}1{]}. ``{[}T{]}here is now \ldots{} {[}a{]}
confidence in the Lord, who will provide'' {[}2{]}. True to his promise,
God provides a ram. Abraham had to learn, if the gospel had to be
fulfilled, only God could bring it about. Abraham names the mountain
``the Lord will provide'' as a tribute

Paul reminds us in \textbf{Ephesians 2} that we are saved though faith
in that it is trusting in God's work that brings about salvation. It is
important to remember this.

\hypertarget{jacob-and-esau}{%
\subsubsection{Jacob and Esau}\label{jacob-and-esau}}

\textbf{Gen 25} Isaac grows up and marries Rebekah who gives birth to 2
sons, Esau and Jacob. The scheming Jacob receives his father's blessing
as though he was not the first born Jacob doesn't deserve that
privilege. He tricks his brother out of his birth right. Jacob means
`holder of the heel'. Twice he held his brother back.

Paul in \textbf{Romans 9} comments on this account ``though they were
not yet born and had done nothing either good or bad \ldots{} not
because of works but because of him who calls''. ``If we are Christians
today, it's not because we are better than anyone else; it's simply
because {[}God chose us{]}'' {[}1{]}. It's important to remember this as
you see the in present time and in history, humans will find any way to
feel inferior to others. We need to learn not to be like that. God later
names Jacob `Israel' which means ``God Strives'' as a reminder it is by
God's effort, not ours.

\hypertarget{gods-rule-and-blessing}{%
\subsection{GOD'S RULE AND BLESSING}\label{gods-rule-and-blessing}}

\hypertarget{gods-law}{%
\subsubsection{God's Law}\label{gods-law}}

You'll' notice that the law is not yet established. Yet, Abraham is
considered righteous. This is an important order. Salvation comes before
the Law. God, through Moses, set up the system of the Law. The Law
signifies God's rule. Following this law will bring blessing.

What God's Law does - reveals our Sin (\textbf{Romans 3}) - reveals his
standards (\textbf{Matthew 5}) - reveals our saviour (\textbf{Galatians
3})

``God's law is not intended to be the means by which anyone gets right
with God \ldots{} He redeems them before they recieve the law \ldots{}
The Law reveals God's standard'' {[}1{]}. This is fortunate as it's
clear early on that Israel are unable to keep this law.

\hypertarget{gods-presence}{%
\subsubsection{God's Presence}\label{gods-presence}}

In Exodus 25 onwards, we learn about the tabernacle. This was a place
where God could set up a sacrificial system to partially clean them
enough to allow Israel to be in his presence, despite their sin.

The writer of the letter in \textbf{Hebrews 10} in teaches us that these
sacrifices never truly took away sin. Leviticus details the ceremonial
acts which needed to take place. This involved lots of animal sacrifice.
Only one man could enter the most Holy place and only once a year. This
was not a pleasant system. Not only that, the Animal sacrifices were
never truly sufficient and it's why they had to do it day after day.

\textbf{Hebrews 10} reminds us as Christians to be thankful we no longer
have to carry out these sacrifices to enjoy fellowship with God.
Christ's blood acted as a once and for all sacrifice.

\hypertarget{gods-place}{%
\subsection{GOD'S PLACE}\label{gods-place}}

Next we have God's place. As we learned from last time, the promised
land was Cannan. \textbf{Josh 21} reveals that God fulfils his promise
of land to Israel by giving them land (but not all of it). They have
also have rest at all 4 sides. God's People are under God's Rule,
enjoying His blessing (`rest'). ``Not one word of all the good promises
that the Lord had made to the house of Israel had failed; all came to
pass''

\hypertarget{gods-king}{%
\subsection{GOD'S KING}\label{gods-king}}

\hypertarget{the-serpent-crusher}{%
\subsubsection{The Serpent crusher}\label{the-serpent-crusher}}

But what of the serpent crusher of \textbf{Gen 3}?\ldots{} Looking back
to \textbf{Gen 49}, Israel knows that it will be one of Judah's
decedents who will be ruler over all nations forever. Israel are still
waiting for this. At this point, God has proven trustworthy with his
promises.

\hypertarget{gods-judges}{%
\subsubsection{God's Judges}\label{gods-judges}}

Israel goes through a series of Judges. In this stage of Israel's
history, they go through a seemingly never-ending cycle where: - Israel
Sins - God Judges by Allowing them to be Defeated - Israel cries for
Mercy - God provides a Judge - Land is at peace - Israel Sins - \ldots{}

At the end of the book of Judges in \textbf{ch.~21} there is a remark
that there was no king in Israel and each done what was right in their
own eyes

\hypertarget{god-rejected-as-king}{%
\subsubsection{God Rejected as King}\label{god-rejected-as-king}}

However, \textbf{1 Sam 8} points out that they did in fact have a king,
God, but Israel had rejected Him and and sought after a human king like
the other nations. If only we had a King like those other nations! They
wanted what the other nations had - a human king to tell them what to do

God anticipates this back in \textbf{Deut 17} and allows them to appoint
a king of his choosing when they reach the promised land. In order for
Israel to receive the promised blessings, this king would have a lot of
responsibility on his shoulders. God rules in his kingdom by means of a
king. The blessings are reliant on that king ``keeping all the words of
this law and these statutes, and doing them''

\hypertarget{gods-king-and-blessing}{%
\subsubsection{God's King and Blessing}\label{gods-king-and-blessing}}

In \textbf{1 Sam}, Samuel, the good judge, appoints Saul as King. Saul
sins and Samuel anoints David as his Saul's successor. \textbf{2 Sam}
David becomes King

In \textbf{2 Sam 7} it's made clear David is not the serpent crusher.
There is still one greater to come who will be a descendent of David.
David's son Solomon takes the throne who partially fulfils this promise.
Solomon builds the temple, a permanent dwelling place for the Lord. The
Temple contains the ark which acts as a symbol of God's Rule. The land
sees prosperity and there is peace at all four sides. Everything looks
good \ldots{} but it does not last. Solomon marries foreign wives and
begins to worship other Gods The kingdom divides. The Partial Kingdom is
in ruins. Turns out Solomon was not the son that was prophesied
about\ldots{}

\hypertarget{bible-study}{%
\subsection{Bible Study}\label{bible-study}}

We will split into groups now. Make sure to fill out your table with
this weeks points, this time under the title `The Partial Kingdom'

\begin{center}\rule{0.5\linewidth}{\linethickness}\end{center}

\hypertarget{refs}{}
\leavevmode\hypertarget{ref-roberts2012}{}%
{[}1{]} V. Roberts, God's Big Picture, IVP, 2012.

\leavevmode\hypertarget{ref-sailhamer2017}{}%
{[}2{]} J.H. Sailhamer, Genesis (The Expositor's Bible Commentary),
2017.
\url{https://www.amazon.co.uk/Genesis-Expositors-Bible-Commentary-Sailhamer-ebook/dp/B01N4SW8K5}
(accessed October 30, 2018).


\end{document}
